\documentclass[journal, table]{IEEEtran}
\usepackage[english, spanish]{babel}
\usepackage[sorting=none]{biblatex}
\bibliography{ref.bib}
\usepackage{amsmath, amsfonts, amsthm}
\usepackage{hyperref, url}
\hyphenation{op-tical net-works semi-conduc-tor}
\usepackage{graphicx}
\usepackage{float}
\usepackage{fancyhdr, last page}
\usepackage{siunitx}
\usepackage{anyfontsize}
\usepackage{csquotes}
\usepackage{svg}
\usepackage{tabularx, ragged2e, booktabs}
\usepackage{xcolor}
\usepackage{multirow}
\usepackage{adjustbox}
\usepackage[affil-sl]{authblk}
\usepackage{tikz}
\usepackage[spanish]{cleveref}
\usepackage{circuitikz}
\usetikzlibrary{patterns}
\usepackage{scalerel}
\usepackage{pict2e}
\usepackage{tkz-euclide}
\usetikzlibrary{calc}
\usetikzlibrary{arrows.meta}
\usetikzlibrary{shadows}
\usetikzlibrary{external}
\usetikzlibrary{decorations.pathmorphing}
\usetikzlibrary{shapes.geometric}
\usetikzlibrary{arrows,shapes.gates.logic.US,shapes.gates.logic.IEC,calc}
\usepackage{pgfplots}
\pgfplotsset{compat=newest}
\usepgfplotslibrary{statistics}
\usepgfplotslibrary{fillbetween}

\AtBeginDocument{\decimalpoint}

\graphicspath{images}
\hypersetup{
    colorlinks=true,
    linkcolor=black,
    urlcolor=blue,
    pdftitle={ALU-JLAB}
}
\urlstyle{same}
\sisetup{separate-uncertainty}

\begin{document}
\tikzstyle{branch}=[fill,shape=circle,minimum size=3pt, inner sep=0pt]

\title{\textbf{Unidad Aritmético-lógica} \\ \small{ALU}}

\author[*]{Julian Avila
    \thanks{Julian Avila: 20212107030}}
\author[*]{Laura Herrera
    \thanks{Laura Herrera: 20212107011}}
\author[*]{Bryan Martínez
    \thanks{Bryan Martínez: 20212107008}}
\author[*]{Juan Acuña
    \thanks{Juan Acuña: 20212107034}}

\affil[*]{Proyecto Curricular de Física\\ Universidad Distrital Francisco José de Caldas}

\date{\today}

\markboth{}
{Shell \MakeLowercase{\textit{et al.}}: Bare Demo of IEEEtran.cls for IEEE Journals}

\maketitle

\section{Objetivos}
\begin{itemize}
    \item Desarrollar una unidad aritmético-lógica para dos números de 4-bits mediante lógica combinacional.
     \item Visualizar usando LEDs el resultado de la correspondiente operación de la ALU.
\end{itemize}

\section{Diseño y solución del problema}
Se plantea realizar una unidad aritmético-lógica (ALU) para números de 4 bits. Para la parte lógica, el circuito realizará las siguientes operaciones:

\begin{align}
A+B \\
A-B \\
\bar{A} \\
A \oplus B \\
\end{align}

En la parte aritmética se realizan las siguientes operaciones:
\begin{align}
A+B \\
A-B \\
A+1 \\
A-1 \\
\end{align}

Para poder generar las diferentes opciones que tiene la ALU se hizo uso de MUXs, de esta forma el usuario puede escoger qué operación desea realizar entre los dos números ingresados.\cite{wakerly-1989}

Para la parte aritmética, se utilizan tres MUX, dos de los cuales tienen la función de escoger si $A$ se va a operar con $B$ o con 1. El tercer MUX tiene la función de escoger si se realizará suma o resta aritmética de los números escogidos.

Para la parte lógica se utilizan también tres MUX, de los cuales dos se encargan de filtrar si se desea hacer las operaciones de suma o resta y $\bar{A}$ o $A \oplus B$, para el tercer MUX, se escoge si se desea ver la parte de la suma y la resta (según se haya escogido) o la parte de $\bar{A}$ y $A \oplus B$. En el caso de las operaciones lógicas, se realizan mediante la comparación de cada bit.

Finalmente, para el resultado que se mostrará en los LEDs, se utiliza un MUX final el cual dará la opción de si se desea obtener la operación aritmética escogida o la lógica.

\section{Diagrama de compuertas Lógicas y Circuito}
\subsection{Sub-unidad Aritmética}
\begin{figure}[h]
    \centering
    \includegraphics[width=\linewidth]{./Images/arithmetische.pdf}
    \caption{Diagrama de la parte aritmética.}
    \label{fig:arithmetic-diagram}
\end{figure}

Como primer sub-unidad se tiene la parte aritmética donde se realizan las operaciones entre dos números de 4 bits que ingresan como señales al sistema. La resta entre $A$ y $B$ o el decrementar $A$ se realiza con la suma del complemento a 2 del valor a restar, por ello se utilizan 2 Full Adder para esta parte y el numero resultado tiene un total de 5-bits donde el de mayor significancia indica el signo del resultado. Para la suma, dependiendo si se desea sumar $A$ con $B$ o solo incrementar $A$, se utiliza un Full Adder, el resultado es un numero de 5-bits mayor a 0, por ello el bit de mayor significancia representa $2^4$.

Ambos resultados son filtrados por un multiplexor que deja el paso de el valor la suma o el valor de la resta.

\subsection{Sub-unidad Logica}
\begin{figure}[h]
    \centering
    \includegraphics[width=\linewidth]{./Images/logik.pdf}
    \caption{Diagrama de la parte lógica.}
    \label{fig:logic-diagram}
\end{figure}

La segunda sub-unidad es la parte lógica, se toman las entradas de 4-bits y sin comparadas por cada operación bit a bit, la señal de salida de la operación pasan a un multiplexor para dar como resultado una de estas.

\subsection{Unidad Aritmético-lógica}
\begin{figure}[H]
    \centering
    \includegraphics[width=\linewidth]{./Images/alu.pdf}
    \caption{Diagrama de la ALU.}
    \label{fig:alu-diagram}
\end{figure}

La unidad completa toma las salidas de las dos sub-unidades y filtra el resultado deseado según el la señal de entrada $S_2$ para ser mostrado por 5 LEDs. La siguiente tabla muestra el resultado mostrado según el estado de las señales de control $S_0, S_1, S_2$:

\begin{table}[htbp!]
    \centering
    \rowcolors{2}{white}{gray!25}
    \begin{tabular}{c|c|c||c}
    \toprule
        $S_2$ & $S_1$ & $S_0$ & $Z$ \\
        \midrule
        0 & 0 & 0 & $A - B$ \\
        0 & 0 & 1 & $A - 1$ \\
        0 & 1 & 0 & $A + B$ \\
        0 & 1 & 1 & $A + 1$ \\
        1 & 0 & 0 & $A \land B$ \\
        1 & 0 & 1 & $A \lor B$ \\
        1 & 1 & 0 & $A \oplus B$ \\
        1 & 1 & 1 & $\Bar{A}$ \\
        \bottomrule
    \end{tabular}
    \caption{Tabla de salidas según las señales de control.}
    \label{tab:Zvalue}
\end{table}

\subsection{Fotos del circuito}
\begin{figure}[h]
    \centering
    \includegraphics[width=\linewidth]{./Images/WhatsApp Image 2024-05-11 at 18.46.47.jpeg}
    \caption{Foto del montaje del circuito.}
    \label{fig:photo}
\end{figure}

\section{Conclusiones}
Se logró el montaje de la ALU mediante compuertas lógicas y MUXs. Esta ALU finalmente muestra la operación seleccionada a través de LEDs, la cual puede ser tanto lógica como aritmética, aplicada a dos números de 4 bits.

\printbibliography
\nocite{*}

\end{document}
