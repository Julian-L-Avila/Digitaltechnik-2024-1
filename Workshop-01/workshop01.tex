\documentclass{article}
\usepackage[utf8]{inputenc}
\usepackage[spanish]{babel}
\usepackage{amsmath, amsfonts, amsthm, amssymb}
\usepackage{hyperref, url}
\usepackage[dvipsnames]{xcolor}
\usepackage{fancyhdr, last page}
\usepackage{siunitx}
\usepackage[margin=2.54cm]{geometry}
\usepackage{graphicx}
\usepackage{subcaption}
\usepackage[affil-sl]{authblk}
\usepackage{anyfontsize}
\usepackage{csquotes}
\usepackage{float}
\usepackage{tabularx, ragged2e, booktabs}
\usepackage{verbatim}
\usepackage[spanish]{cleveref}

\newenvironment{problem}[2][Problema]{\begin{trivlist}
\item[\hskip \labelsep {\bfseries #1}\hskip \labelsep {\bfseries #2.}]}{\end{trivlist}}

\newenvironment{solution}[1][Solución]{\begin{trivlist}
\item[\hskip \labelsep {\bfseries #1:}]}{\end{trivlist}}


\hypersetup{
    colorlinks=true,
    linkcolor=black,
    urlcolor=blue,
    pdftitle={Workshop01-Digitaltechnik}
}
\urlstyle{same}
\renewcommand\Affilfont{\fontsize{8}{8}}
\makeatletter
\renewcommand\AB@affilsepx{, \protect\Affilfont}
\makeatother

\title{\textbf{Taller 01\\ \small{Electrónica Digital}}}

\author[1]{Julian Avila}
\affil[1]{20212107030}
\date{3 de Abril 2024}

\pagestyle{fancy}
\fancyhead{}
\fancyfoot{}
\lhead{Taller 01 - Electrónica Digital}
\rhead{Julian Avila}
\rfoot{Pagina \thepage\ de \pageref{LastPage}}

\begin{document}

\maketitle
\thispagestyle{fancy}
\hrule

\begin{problem}{1}
Dada la siguiente igualdad:

$$(100)_{10} = (400)_{b}$$

Determinar el valor de la base $b$ y el valor de $(104)_{10}$ en la base $b$.
\end{problem}

\begin{solution}
\textbf{a.}
\begin{equation*}
    10^{2} = 4b^2
\end{equation*}
\begin{equation*}
    \boxed{
        \therefore b = 5
    }
\end{equation*}

\textbf{b.}
\begin{align*}
    \frac{104}{5} & = 20(5) + 4 \\
    \frac{20}{5}  & = 4(5) \\
    \implies (104)_{10}    & = 4(5)^{2} + 0(5) + 4(5)^0
\end{align*}

\begin{equation*}
    \boxed{
        \therefore (104)_{10} = (404)_{5}
    }
\end{equation*}
\end{solution}


\begin{problem}{2}
Dado la siguiente igualdad:

$$ (174)_{10} = (450)_{b} $$

Determinar el valor de la base $b$ y el valor de $(104)_{10}$ en la base $b$.
\end{problem}

\begin{solution}
\textbf{a.}
\begin{align*}
    10^{2} + 7(10) + 4                 & = 4b^{2} + 5b \\
    b^{2} + \frac{5}{4}b               & = \frac{174}{4} \\
    \left( b + \frac{5}{8} \right)^{2} & = \frac{2809}{64} \\
    b + \frac{5}{8}                    & = \frac{53}{8} \\
\end{align*}

\begin{equation*}
    \boxed{
        \therefore b = 6
    }
\end{equation*}

\textbf{b.}
\begin{align*}
    \frac{104}{6}       & = 17(6) + 2 \\
    \frac{17}{6}        & = 2(6) + 5 \\
    \implies (104)_{10} & = 2(6)^{2} + 5(6) + 2
\end{align*}

\begin{equation*}
    \boxed{
        \therefore (104)_{10} = (252)_{6}
    }
\end{equation*}
\end{solution}


\begin{problem}{3}
Decodificar el siguiente mensaje codificado en ASCII y escribir el mensaje en hexadecimal.

\begin{align*}
    &\texttt{1001000 1100101 1101100 1101100 1101111 0101110 0100000 1001000 1101111 1110111} \\ 
    &\texttt{0100000 1100001 1110010 1100101 0100000 1111001 1101111 1110101 0111111}
\end{align*}
\end{problem}

\begin{solution}{\textbf{a} Decodificación:}
\begin{itemize}
    \item $(1001000)_{2} = 2^{6} + 2^{3} = (72)_{10} = \text{H}$ 
    \item $(1100101)_{2} = 2^{6} + 2^{5} + 2^{2} + 1 = (101)_{10} = \text{e}$
    \item $(1101100)_{2} = 2^{6} + 2^{5} + 2^{3} + 2^{2} = (108)_{10} = \text{l}$
    \item $(1101111)_{2} = 2^{6} + 2^{5} + 2^{3} + 2^{2} + 2 + 1 = (111)_{10} = \text{o}$
    \item $(0101110)_{2} = 2^{5} + 2^{3} + 2^{2} + 2 = (46)_{10} = \text{.}$
    \item $(0100000)_{2} = 2^{5} = (32)_{10} = \text{ (space)}$
    \item $(1110111)_{2} = 2^{6} + 2^{5} + 2^{4} + 2^{2} + 2 + 1 = (119)_{10} = \text{w}$
    \item $(1100001)_{2} = 2^{6} + 2^{5} + 1 = (97)_{10} = \text{a}$
    \item $(1110010)_{2} = 2^{6} + 2^{5} + 2^{4} + 2 = (114)_{10} = \text{r}$
    \item $(1111001)_{2} = 2^{6} + 2^{5} + 2^{4} + 2^{3} + 1 = (121)_{10} = \text{y}$
    \item $(1110101)_{2} = 2^{6} + 2^{5} + 2^{4} + 2^{2} + 1 = (117)_{10} = \text{u}$
    \item $(0111111)_{2} = 2^{5} + 2^{4} + 2^{3} + 2^{2} + 1 = (63)_{10} = \text{?}$
\end{itemize}

\begin{center}
\framebox[1.1\width]{Mensaje: Hello. How are you?}
\end{center}

\text{b.} Codificación hexadecimal:
\begin{itemize}
    \item  $ (72)_{10}  = 4(16) + 8 = (48)_{16}$
    \item  $ (101)_{10} = 6(16) + 5 = (65)_{16}$
    \item  $ (108)_{10} = 6(16) + 12 = (6C)_{16}$
    \item  $ (111)_{10} = 6(16) + 15 = (6F)_{16}$
    \item  $ (46)_{10}  = 2(16) + 14 = (2E)_{16}$
    \item  $ (32)_{10}  = 2(16) = (20)_{16}$
    \item  $ (119)_{10} = 7(16) + 7 = (77)_{16}$
    \item  $ (97)_{10}  = 6(16) + 1 = (61)_{16}$
    \item  $ (114)_{10} = 7(16) + 2 = (72)_{16}$
    \item  $ (121)_{10} = 7(16) + 9 = (79)_{16}$
    \item  $ (117)_{10} = 7(16) + 5 = (75)_{16}$
    \item  $ (63)_{10}  = 3(16) + 15 = (3F)_{16}$
\end{itemize}

\begin{center}
\framebox[1.1\width]{Mensaje (Hexadecimal): 48 65 6C 6C 6F 2E 20 48 6F 77 20 61 72 65 20 79 6F 75 3F}
\end{center}
\end{solution}


\begin{problem}{4}
    Hallar analíticamente el valor de $b$ que satisfaga la igualdad. Las operaciones aritméticas deben ser realizadas es una base diferente a decimal.
    
    $$(179)_{b} = (666)_{7} + (1023)_{4} $$
\end{problem}

\begin{solution}
    \begin{align*}
        (1023)_{4} &= 4^{3} + 2(4) + 3\\
         &= (75)_{10}  \\
         &= 7^2 + 3(7) + 5 \\
         &= (135)_{7}
    \end{align*}

    \begin{align*}
        (179)_{b} &= (666)_7 + (135)_{7} \\
        &= (1134)_7 \\
        b^{2} + 7b + 9 &= 7^{3} + 7^{2} + 3(7) + 4 \\
        b^{2} + 7b &= 408 \\
        \left(b + \frac{7}{2}\right)^{2} &= \frac{1681}{4} \\
        b + \frac{7}{2} &= \frac{41}{2} 
    \end{align*}

    \begin{equation*}
        \boxed{
            \therefore b = 17
        }
    \end{equation*}
\end{solution}

\end{document}
